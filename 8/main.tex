\input{../header.tex}

\title{Blatt 8}
\date{
  Abgabe: 28.06.2018
}

%\DeclareMathOperator{\Tr}{Tr}

\begin{document}
\parindent0mm
\section*{Aufgabe 1}

\subsection*{a)}
\begin{align}
\intertext{Kinetische Energie:}
T &= \frac{1}{2}m_1\dot{\theta}^2 L_1^2 + \frac{1}{2}\left(m_2 \dot{\theta_2}^2 L_2 + m_2 \dot{\theta_1}^2 L_1 + 2 m_2 \dot{\theta_1}^2 L_1 \dot{\theta_2}^2 L_2 \cos(\theta_1 -\theta_2)\right)
\intertext{Potentielle Energie:}
V &= -m_1g_L1 \cos(\theta_1)-m_2gL_1 \cos(\theta_1) -m_2gL_2\cos(\theta_2)
\end{align}
\subsection*{b)}
Nährung liefert:

\begin{align}
 \ddot{\theta}_1 &= \frac{1}{1-\mu}\left( \mu g_1 \theta_2 -g_1 \theta_1\right)\\
 \ddot{\theta}_2 &= \frac{1}{1-\mu}\left( g_2 \theta_1 -g_2\theta_2\right)
\intertext{In Matrixform:}
\begin{pmatrix}
 \ddot{\theta}_1\\
 \ddot{\theta}_2
\end{pmatrix}
&= \frac{1}{1-\mu}
\begin{pmatrix}
-g_1    & \mu g_1 \\
 g_2    &    -g_2
\end{pmatrix}
\begin{pmatrix}
\theta_1\\
 \theta_2
\end{pmatrix}
\end{align}

\subsection*{c/d)}
Siehe \texttt{anim.mp4}. Die angegebenen Startbedingungen waren uns zu öde, daher haben wir spannendere gewählt.
%
% \begin{align}
% \intertext{mit den Eigenwerten $\lambda$ für $\mu=\frac{1}{2}$:}
% \lambda_{1/2} =  - \left(g_1+g_2\right)\pm\sqrt{g_1^2+g_2^2}
% \intertext{somit folgt für die DGLn:}
%  \ddot{\theta}_1 -\lambda_1 \theta_1 &=0\\
%  \ddot{\theta}_2 -\lambda_2 \theta_2 &=0
% \intertext{mit den Lösungen:}
% \theta_1(t) = a_1 \cos(\sqrt{-\lambda_1} t)
% \theta_2(t) = a_2 \cos(\sqrt{-\lambda_2} t)
% \end{pmatrix}
% \end{align}
%
%





\section*{Aufgabe 2}
\includegraphics[width=0.9\textwidth]{build/poincaré_+.txt.pdf}\\
\includegraphics[width=0.9\textwidth]{build/poincaré_chaos.txt.pdf}\\
\includegraphics[width=0.9\textwidth]{build/poincaré_quasi.txt.pdf}\\
\includegraphics[width=0.9\textwidth]{build/euclid.pdf}
\end{document}
