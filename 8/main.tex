\documentclass[
  bibliography=totoc,     % Literatur im Inhaltsverzeichnis
  captions=tableheading,  % Tabellenüberschriften
  titlepage=firstiscover, % Titelseite ist Deckblatt
  DIV=18
]{scrartcl}

% Paket float verbessern
\usepackage{scrhack}

% Seitenränder etc.
%\usepackage[a4paper]{geometry}

% Warnung, falls nochmal kompiliert werden muss
\usepackage[aux]{rerunfilecheck}

% deutsche Spracheinstellungen
\usepackage{polyglossia}
\setmainlanguage{german}

\usepackage{sourceserifpro}
\usepackage{sourcesanspro}
%\usepackage{charter}

% unverzichtbare Mathe-Befehle
\usepackage{amsmath}
% viele Mathe-Symbole
\usepackage{amssymb}
% Erweiterungen für amsmath
\usepackage{mathtools}


% Fonteinstellungen
\usepackage{fontspec}
% Latin Modern Fonts werden automatisch geladen

\usepackage[
  math-style=ISO,    % ┐
  bold-style=ISO,    % │
  sans-style=italic, % │ ISO-Standard folgen
  nabla=upright,     % │
  partial=upright,   % ┘
  warnings-off={           % ┐
    mathtools-colon,       % │ unnötige Warnungen ausschalten
    mathtools-overbracket, % │
  },                       % ┘
]{unicode-math}


% traditionelle Fonts für Mathematik
%\setmainfont{SourceSerifPro}
%\setmathfont{Latin Modern Math}
%\setmathfont{XITS Math}[range={scr, bfscr}]
%\setmathfont{XITS Math}[range={cal, bfcal}, StylisticSet=1]

% Zahlen und Einheiten
\usepackage[
  locale=DE,                 % deutsche Einstellungen
  separate-uncertainty=true, % immer Fehler mit \pm
  per-mode=reciprocal,       % ^-1 für inverse Einheiten
%  output-decimal-marker=.,   % . statt , für Dezimalzahlen
]{siunitx}

% chemische Formeln
\usepackage[
  version=4,
  math-greek=default, % ┐ mit unicode-math zusammenarbeiten
  text-greek=default, % ┘
]{mhchem}

% richtige Anführungszeichen
\usepackage[autostyle]{csquotes}

% schöne Brüche im Text
\usepackage{xfrac}

% Standardplatzierung für Floats einstellen
\usepackage{float}
\floatplacement{figure}{htbp}
\floatplacement{table}{htbp}

% Floats innerhalb einer Section halten
\usepackage[
  section, % Floats innerhalb der Section halten
  below,   % unterhalb der Section aber auf der selben Seite ist ok
]{placeins}

% Captions schöner machen.
\usepackage[
  labelfont=bf,        % Tabelle x: Abbildung y: ist jetzt fett
  font=small,          % Schrift etwas kleiner als Dokument
  width=0.9\textwidth, % maximale Breite einer Caption schmaler
]{caption}
% subfigure, subtable, subref
\usepackage{subcaption}

% Grafiken können eingebunden werden
\usepackage{graphicx}
% größere Variation von Dateinamen möglich
\usepackage{grffile}

% schöne Tabellen
\usepackage{booktabs}

% Verbesserungen am Schriftbild
\usepackage{microtype}

% Literaturverzeichnis
%\usepackage[
%  backend=biber,
%  sorting=none
%]{biblatex}

% Hyperlinks im Dokument
\usepackage[
  unicode,        % Unicode in PDF-Attributen erlauben
  pdfusetitle,    % Titel, Autoren und Datum als PDF-Attribute
  pdfcreator={},  % ┐ PDF-Attribute säubern
  pdfproducer={}, % ┘
]{hyperref}
% erweiterte Bookmarks im PDF
\usepackage{bookmark}

% Trennung von Wörtern mit Strichen
\usepackage[shortcuts]{extdash}

\usepackage[shortlabels]{enumitem}

\usepackage{braket}

\DeclareMathOperator{\Tr}{Tr}
\usepackage[makeroom]{cancel}

%Eigene Befehle
\ExplSyntaxOn
\NewDocumentCommand \fig {mmmO{}}
{
  \begin{figure}
    \centering
    \includegraphics[#4]{#1}
    \caption{#2}
    \label{fig:#3}
  \end{figure}
}
\ExplSyntaxOff


\author{Dag-Björn Hering\\Lars Funke}
\subtitle{Computational Physics}


\title{Blatt 8}
\date{
  Abgabe: 28.06.2018
}

%\DeclareMathOperator{\Tr}{Tr}

\begin{document}
\parindent0mm
\section*{Aufgabe 1}

\subsection*{a)}
\begin{align}
\intertext{Kinetische Energie:}
T &= \frac{1}{2}m_1\dot{\theta}^2 L_1^2 + \frac{1}{2}\left(m_2 \dot{\theta_2}^2 L_2 + m_2 \dot{\theta_1}^2 L_1 + 2 m_2 \dot{\theta_1}^2 L_1 \dot{\theta_2}^2 L_2 \cos(\theta_1 -\theta_2)\right)
\intertext{Potentielle Energie:}
V &= -m_1g_L1 \cos(\theta_1)-m_2gL_1 \cos(\theta_1) -m_2gL_2\cos(\theta_2)
\end{align}
\subsection*{b)}
Nährung liefert:

\begin{align}
 \ddot{\theta}_1 &= \frac{1}{1-\mu}\left( \mu g_1 \theta_2 -g_1 \theta_1\right)\\
 \ddot{\theta}_2 &= \frac{1}{1-\mu}\left( g_2 \theta_1 -g_2\theta_2\right)
\intertext{In Matrixform:}
\begin{pmatrix}
 \ddot{\theta}_1\\
 \ddot{\theta}_2
\end{pmatrix}
&= \frac{1}{1-\mu}
\begin{pmatrix}
-g_1    & \mu g_1 \\
 g_2    &    -g_2
\end{pmatrix}
\begin{pmatrix}
\theta_1\\
 \theta_2
\end{pmatrix}
\end{align}

\subsection*{c/d)}
Siehe \texttt{anim.mp4}. Die angegebenen Startbedingungen waren uns zu öde, daher haben wir spannendere gewählt.
%
% \begin{align}
% \intertext{mit den Eigenwerten $\lambda$ für $\mu=\frac{1}{2}$:}
% \lambda_{1/2} =  - \left(g_1+g_2\right)\pm\sqrt{g_1^2+g_2^2}
% \intertext{somit folgt für die DGLn:}
%  \ddot{\theta}_1 -\lambda_1 \theta_1 &=0\\
%  \ddot{\theta}_2 -\lambda_2 \theta_2 &=0
% \intertext{mit den Lösungen:}
% \theta_1(t) = a_1 \cos(\sqrt{-\lambda_1} t)
% \theta_2(t) = a_2 \cos(\sqrt{-\lambda_2} t)
% \end{pmatrix}
% \end{align}
%
%





\section*{Aufgabe 2}
\includegraphics[width=0.9\textwidth]{build/poincaré_+.txt.pdf}\\
\includegraphics[width=0.9\textwidth]{build/poincaré_chaos.txt.pdf}\\
\includegraphics[width=0.9\textwidth]{build/poincaré_quasi.txt.pdf}\\
\includegraphics[width=0.9\textwidth]{build/euclid.pdf}
\end{document}
