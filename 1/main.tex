\input{../header.tex}

\title{Blatt 1}
\date{
  Abgabe: 20.04.2018
}


\begin{document}
\maketitle

\section*{Aufgabe 1}
\subsection*{a)}
% \begin{figure}
%   \includegraphics{build/1.pdf}
%   \caption{Plot für $R_N(N)$.}
% \end{figure}
\subsection*{b)}
Damit sich eine Gerade ergbit, muss $R_N$ gegen $N^{\frac{1}{4}}$ aufgetragen werden.

\section*{Aufgabe 2}

\subsection*{b)}
Für $T_1=-1$  verwende Substitution $x=\frac{1}{u}$:
\begin{align*}
\Rightarrow -\int_0^{\frac{1}{T}} \frac{1}{\sqrt{\pi}}\frac{1}{u^2}\exp{-\frac{1}{u^2}}\symup{d} u
\end{align*}
Für $T_2=\num{1.1631}$ und $T_3=\infty$  verwende Substitution $x=\tan\left(u\right)$:
\begin{align*}
\Rightarrow \int_{\frac{-\pi}{2}}^{\arctan(T)} \frac{1}{\sqrt{\pi}}\frac{1}{\cos(x)^2}\exp{-\tan\left(u\right)^2}\symup{d} u
\end{align*}
\end{document}
