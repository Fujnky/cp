\input{../header.tex}

\title{Blatt 2}
\date{
  Abgabe: 04.05.2018
}


\begin{document}
%\maketitle

\section*{Aufgabe 1}
\subsection*{b)}
\fig{build/1_hist.pdf}{(i)}{}[width=\textwidth]
\fig{build/2_hist.pdf}{(ii)}{}[width=\textwidth]
\fig{build/3_hist.pdf}{(iii)}{}[width=\textwidth]
\fig{build/4_hist.pdf}{(iv)}{}[width=\textwidth]
\FloatBarrier
\subsection*{c)}
\fig{build/1_scatter.pdf}{(i)}{}[width=\textwidth]
\fig{build/2_scatter.pdf}{(ii)}{}[width=\textwidth]
\fig{build/3_scatter.pdf}{(iii)}{}[width=\textwidth]
\fig{build/4_scatter.pdf}{(iv)}{}[width=\textwidth]
\FloatBarrier
\subsection*{d)}
\fig{build/5_hist.pdf}{(11,1,7)}{}[width=\textwidth]
\fig{build/6_hist.pdf}{(11,4,7)}{}[width=\textwidth]
\fig{build/5_scatter.pdf}{(11,1,7)}{}[width=\textwidth]
\fig{build/6_scatter.pdf}{(11,4,7)}{}[width=\textwidth]
\FloatBarrier
\subsection*{e)}
\fig{build/period.pdf}{}{}[width=\textwidth]

\section*{Aufgabe 2}
\subsection*{a)}
Die Zufallsvariable $Y$ muss im Intervall $[0,\frac{\pi}{2}]$ liegen,
damit $P(y)=\sin(y)$ die zugehörige Verteilungsfunktion
von der Dichtefunktion $p(y)=\cos(y)$ ist.
\subsubsection*{i)}
Transformationsmethode für Gleichverteilung U(x) im Intervall $[0,1]$
\begin{align}
  \int_0^u U(x)dx&=\int_0^y \cos(y')\symup{d}y'\\
  u&=\sin(y)-\sin(0)
  \arcsin(u)&=y
\end{align}
Einsetzen der Gleichverteiltenzahlen in $\arcsin(u)$
liefert $\cos$-verteilte Zufallszahlen.

\begin{figure}
  \includegraphics{build/a2_hist_1.pdf}
  \caption{Histogramm für i).}
\end{figure}

\subsubsection*{ii)}
Transformationsmethode für Gleichverteilung U(x) im Intervall $[-1,1]$
\begin{align}
\frac{1}{2}\int_-1^u U(x)dx&=\int_0^y \cos(y')\symup{d}y'\\
\frac{1}{2}(u+1)&=\sin(y)-\sin(0)
\arcsin\left(\frac{1}{2}(u+1)\right)&=y
\end{align}
Einsetzen der Gleichverteiltenzahlen in $\arcsin\left(\frac{1}{2}(u+1)\right)$
liefert $\cos$-verteilte Zufallszahlen.

\begin{figure}
  \includegraphics{build/a2_hist_2.pdf}
  \caption{Histogramm für ii).}
\end{figure}

$\rightarrow$ Beide Methoden liefer gleiche Ergbnisse, Wobei die ii) umständicher
ist, da das Intergral auf der Linkenseite zunächst berechnet werden muss vor die Inversefunktion
berechnet werden kann.


\subsection*{b)}
Es werden zwei gleichverteilte Zufallszahlen $u_1,u_2$ im Intervall $[0,1]$
benötigt und mit der Vorschrift
\begin{align}
  z_1=\sqrt{2\ln u_1}\cos(2\pi u_2)
  z_2=\sqrt{2\ln u_1}\sin(2\pi u_2)
\end{align}
werden zwei standardnormalverteilte
Zufallszahlen erzeugt.
Um nun beliebige normalverteilte
Zufallszahlen mit $\mu$ und $\sigma$ zu erzeugen
wird die Formel
\begin{align*}
x_i= \mu +\sigma\cdot z_i
\end{align*}
verwendet.

\begin{figure}
  \includegraphics{build/a2_hist_3.pdf}
  \caption{Histogramm von normalverteilten Zufallszahlen mit $\mu=3$ und $\sigma^2=4$.}
\end{figure}
\end{document}

\subsection*{c)}
